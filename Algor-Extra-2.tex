\documentclass[9pt]{beamer}

% Presento style file
\usepackage{config/presento}

% custom command and packages
\input{config/custom-command}
\usepackage{listings}

\usepackage{amsopn} %eps stuff
\usepackage[dvips]{epsfig} %eps stuff
\usepackage[utf8]{inputenc}
\usepackage[sc]{mathpazo}



\usepackage[sc]{mathpazo}
\linespread{1.05}         % Palladio needs more leading (space between lines)
\usepackage[T1]{fontenc}

\usepackage{cmap}
\usepackage{mathdots}
\usepackage{microtype}
\usepackage[noadjust]{cite}
\usepackage{ulem}
\usepackage{lipsum}
\usepackage{enumerate}
\usepackage{multicol}
\usepackage{amsthm}
\usepackage{enumitem}
\usepackage{graphicx}






% Information
\title{Home Work 4 - Problem 5}
\subtitle{Algorithms}
\author{Rory Flynn, Balaviknesh Sekar}
\institute{University of Colorado Denver}
\date{\today}


\usepackage{Sweave}
\begin{document}
\Sconcordance{concordance:Algor-Extra-2.tex:Algor-Extra-2.Rnw:%
1 45 1 1 %
0 1 1 1 %
2 1 0 1 %
2 1 0 1 %
1 1 6 7 %
0 1 2 12 %
1 1 4 3 %
0 1 1 3 %
0 1 2 72 %
1}

\begin{Schunk}
\begin{Sinput}
> library(knitr)
> #options (replace.assign = tRUE)
> render_listings()
> library(igraph)
> #Sys.setenv(RSTUDIO_PDFLATEX = "LuaLaTeX")
> #set global chunk options
> #opts_chunk$set(fig.align='center',fig.show='hold',highlight=FALSE)
> #opts_chunk$set(fig.width=.5, fig.height=.5)
> options(formatR.arrow=TRUE,width=10)
\end{Sinput}
\end{Schunk}
% Title page
\begin{frame}[plain]
\maketitle
\end{frame}





\begin{frame}[fragile]{Home Work 4: Problem 5: Thickness}
     Let $G$ be a simple undirected graph with n vertices labeled $1,2,. . .,n$. The graph $G[r_1 , r_2 , . . . , r_n ]$ is the graph obtained from G by replacing vertex $i$ with $K_{r_i}$ and connecting all possible vertices in neighboring complete graphs. So, for example, $G[1, 1, . . . , 1] = G$, $K_2 [2, 2] = K_4$ , $K_2 [m, n] = K_{m+n}$ , and $K_1 [n] = K_n$.
   
  Using R and igraph:
\begin{Schunk}
\begin{Sinput}
> g <- make_empty_graph(directed=F ) +
+   vertices(1:3) +
+   edge( c(1:3,1:3))
>   plot(g)
\end{Sinput}
\end{Schunk}
  
\end{frame}

\section{Section no. 1}
\begin{frame}{Part a }
    \begin{problem}
    The independence number $\alpha(G)$ of a graph $G$ is the size of the largest set of independent (mutually nonadjacent) vertices in G. Prove that $\chi(G) \ge \frac{|V (G)|}{\alpha(G)}$.
     \end{problem}
    \begin{proof}
        Let $a_1$ be the largest set of independent (mutually nonadjacent) vertices in G.

        Then there is a set $!a_1$ the members of which are by definition not in $a_1$ but are agent to a member of $a_1$.

        The set $!a_1$ can be sub divided in to set of independent vertices none of wich can be lager than $a_1$ or smaller than 1.

        It fallows that the maximum number of subsets is then $\frac{|V (G)|}{\alpha(G)}$.

        Note that for a correct coloring all members of an independent set can share one color(they are not adjacent), no two sets can share a color however (their nodes are adjacent).

        Thus the chromatic number must be greater or equal the number of independent sub sets.

        Therefore $\chi(G) \ge \frac{|V (G)|}{\alpha(G)}$.
    \end{proof}
\end{frame}

\section{Section no. 2}
\frame{\frametitle{Part b}
    If $G$ is a graph with $n$ vertices, prove that $\alpha(G) = \alpha(G[r_1 , r_2, ... , r_n])$, where each $r_i \in N$.
    \begin{proof}
    
$\alpha (G)$ can be weird as the sum of independence number of the sub graphs of (G) 

Each vertex in the largest independent set of G obviously only adds one to $\alpha (G)$

Recall that each $r_i$ replaces a original point in (G) and forms a complete graph with its adjacent vertices.

The maximum independence number of the complete graph created by the replacement of the vertices with the $r_i$ is one.

Therefore the subgraph will contribute the same value to $\alpha(G[r_1 , r_2, ... , r_n])$ as the original point contributed to $\alpha (G)$ which is 1.


Thus $\alpha(G) = \alpha(G[r_1 , r_2, ... , r_n])$

    \end{proof}

}

\frame{\frametitle{Part c}
    Find both the chromatic number and thickness of $C_3 [2, 2, 2]$ and prove that your answers are correct.
}

\frame{\frametitle{Part d}
    Find both the chromatic number and thickness of $C_5 [2, 2, 2, 2, 2]$ and prove that your answers are correct.
}

\frame{\frametitle{Part e}
    Find both the chromatic number and thickness of $C_n [2, 2, . . . , 2]$ and prove that your answers are correct.
}

\frame{\frametitle{Part f}
    Find both the chromatic number and thickness of $C_5 [3, 3, 3, 3, 3]$ and prove that your answers are correct.
}

\frame{\frametitle{Part g}
    Find both the chromatic number and thickness of $C_5 [4, 4, 4, 4, 3]$ and prove that your answers are correct.
}

\frame{\frametitle{Part h}
    Find both the chromatic number and thickness of $C_7 [4, 4, 4, 4, 4, 4, 4]$ and prove that your answers are correct.
}


\end{document}
